\documentclass{../lista}

\begin{document}
	\cabecalhoAlt
		{Simulado | 2º Intensivo para a OBA}
		{\red{Gabarito}}
		{Gabriel Lucena e Iago Mendes}

	\red{Observação: \begin{itemize}
		\item As alternativas das perguntas deste gabarito não estão na mesma ordem do simulado.
	\end{itemize}}

	\begin{secao}{Questões de Astronomia}
	\end{secao}

	\begin{secao}{Questões de Astronáutica}
	\end{secao}

	\begin{secao}{Questões Avançadas}
		\begin{questao}{(1 ponto)}
			Um fenômeno muito conhecido é o da ``laçada de Marte", em que o planeta Marte subitamente muda sua direção de deslocamento no céu, e quando acompanhado por vários dias parece se locomover formando um laço no céu.

			\begin{pergunta}{(1 ponto)}
				Quais planetas, além de Marte, reproduzem o mesmo fenômeno de modo que possamos observá-los em uma noite de céu limpo?

				\red{\begin{itemize}
						\item Todos os planetas reproduzem esse fenômeno. Então, a pegadinha da questão é você marcar somente os planetas que são observáveis à noite, excluindo assim os planetas inferiores (Mercúrio e Vênus), os quais estão sempre próximos ao Sol na Esfera Celeste.
				\end{itemize}}

				\begin{multicols}{3}
					\begin{alternativas}
						\item[$(\quad)$] Mercúrio
						\item[$(\quad)$] Vênus
						\alternativaMarcada Júpiter
						\alternativaMarcada Saturno
						\alternativaMarcada Urano
						\alternativaMarcada Netuno
					\end{alternativas}
				\end{multicols}
			\end{pergunta}
		\end{questao}
	\end{secao}
\end{document}